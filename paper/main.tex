%%%%%%%% ICML 2021 EXAMPLE LATEX SUBMISSION FILE %%%%%%%%%%%%%%%%%

\documentclass{article}

% Recommended, but optional, packages for figures and better typesetting:
\usepackage{microtype}
\usepackage{graphicx}
%\usepackage{subfigure}
\usepackage{booktabs} % for professional tables
\usepackage{amsmath,amssymb}

% hyperref makes hyperlinks in the resulting PDF.
\usepackage{hyperref}

% Attempt to make hyperref and algorithmic work together better:
\newcommand{\theHalgorithm}{\arabic{algorithm}}

% Use the following line for the initial blind version submitted for review:
\usepackage[accepted]{comp541}

% The \icmltitle you define below is probably too long as a header.
% Therefore, a short form for the running title is supplied here:
\icmltitlerunning{GPW-SP: Gaussian Pancakes Watermarking with Salted Phase}

\begin{document}

\twocolumn[
\icmltitle{GPW-SP: Gaussian Pancakes Watermarking with Salted Phase}

% List of affiliations: The first argument should be a (short)
% identifier you will use later to specify author affiliations
% Academic affiliations should list Department, University, City, Region, Country
% Industry affiliations should list Company, City, Region, Country

\icmlsetsymbol{equal}{*}

\begin{icmlauthorlist}
\icmlauthor{Halil İbrahim Kanpak}{equal,to}
\icmlauthor{Mehmet Kantar}{equal,to}
\icmlauthor{Melih Okşak}{equal,to}
\end{icmlauthorlist}

\icmlaffiliation{to}{College of Engineering, Koç University, Istanbul, Turkey}

\icmlcorrespondingauthor{Halil İbrahim Kanpak}{hkanpak21@ku.edu.tr}
\icmlcorrespondingauthor{Mehmet Kantar}{mkantar21@ku.edu.tr}
\icmlcorrespondingauthor{Melih Okşak}{moksak20@ku.edu.tr}

% You may provide any keywords that you
% find helpful for describing your paper; these are used to populate
% the "keywords" metadata in the PDF but will not be shown in the document
\icmlkeywords{Watermarking, LLM, GPW-SP, Gaussian Pancakes}

\vskip 0.3in
]

% this must go after the closing bracket ] following \twocolumn[ ...

% This command actually creates the footnote in the first column
% listing the affiliations and the copyright notice.
% The command takes one argument, which is text to display at the start of the footnote.
% The \icmlEqualContribution command is standard text for equal contribution.
% Remove it (just {}) if you do not need this facility.

%\printAffiliationsAndNotice{}  % leave blank if no need to mention equal contribution
\printAffiliationsAndNotice{\icmlEqualContribution} % otherwise use the standard text.

\begin{abstract}
We study private watermarking for autoregressive language models, where a provider embeds a key-dependent statistical signal during decoding and later verifies provenance using a secret key.
We introduce \emph{Gaussian Pancakes Watermarking with Salted Phase} (GPW-SP), a lightweight sampling-time method that biases token selection via a smooth periodic function of token embedding projections onto a secret direction.
To reduce predictability, GPW-SP incorporates a context-dependent salted phase derived from a keyed pseudorandom function, producing position-varying preferences without modifying model weights.

GPW-SP is implemented as a modular logits-bias component compatible with standard decoding schemes and evaluated using calibrated hypothesis testing at low false positive rates.
In a GPT-2 pilot on C4 prompts, phase salting substantially improves clean detectability over an unsalted variant (AUC $0.964$ vs.\ $0.878$, TPR@1\%FPR $0.70$ vs.\ $0.28$).
Under meaning-preserving edits, detectability degrades smoothly, remaining robust to moderate synonym replacement and word deletion, while stronger paraphrasing and text mixing significantly reduce detection power.

Overall, GPW-SP provides an efficient geometry-based watermark baseline and a reproducible evaluation protocol aligned with recent watermarking benchmarks, reporting robustness curves under transformations.
\end{abstract}

\section{Introduction}
\label{sec:introduction}


While Large Language Models (LLMs) enhance productivity, they also facilitate spam, plagiarism, and disinformation. This necessitates detection of machine-written text and/or attribution of that text to a specific model. Unlike brittle post-hoc classifiers that fail under paraphrasing or domain shift \cite{zhao2025sokwatermarkingaigeneratedcontent}, \emph{watermarking} embeds a signal during generation by modifying sampling behavior . With a secret key, a verifier can detect or or attribute this signal to a source without model retraining, often via a modular logits processor.


Designing effective watermarks is challenging because the channel is adversarial and stochastic. Text is easily edited through paraphrasing or truncation, and decoding methods like nucleus sampling introduce randomness. Consequently, watermarking creates a \emph{statistical bias} rather than a deterministic signature, framing detection as a hypothesis test. This creates a fundamental tension between \emph{utility} (text quality), \emph{detectability} (true positive rate), and \emph{robustness} (resistance to transformation). 
Recent work has also emphasized that watermark conclusions can vary significantly across prompt sets, decoding parameters, and transformation attacks.
This motivates benchmark-driven evaluation: reporting not only ``clean'' detectability, but robustness under realistic editing pipelines and model-based attacks, ideally as curves over attack strength using standardized suites and tooling (e.g., WaterBench, WaterPark, MarkLLM).

\paragraph{Our approach: Gaussian Pancakes Watermarking with Salted Phase (GPW-SP).}
We propose GPW-SP, a watermark tied to the geometry of the model's token embedding space. It projects embeddings onto a secret keyed direction $w$ and applies a periodic cosine score to bias logits. To ensure unpredictability, we use a \emph{salted phase}---a context-dependent shift derived from a pseudorandom function. 

\paragraph {Our contributions}: (i) GPW-SP, a simple embedding-geometry watermark
that requires no retraining and integrates as a logits-bias module; (ii) a private detection procedure
framed as a calibrated hypothesis test with user-chosen false positive rates; (iii) optional extensions for
semantic coupling (SR) and payload encoding; and (iv) an evaluation protocol with robustness curves
under common text transformations, plus practical stress tests such as deletion and splicing/mixing.

\section{Background and Problem Setup}
\label{sec:background}

\subsection{Watermarking for autoregressive LMs (private provenance)}
We consider an autoregressive language model that, given a prefix $x_{1:t-1}$, outputs logits
$\ell_t \in \mathbb{R}^{|\mathcal{V}|}$ and induces a next-token distribution
$p_t = \mathrm{softmax}(\ell_t/\tau)$ (temperature $\tau$).
A \emph{watermarked sampler} modifies \emph{inference-time} sampling—most commonly by adding a
key-dependent bias $b_t \in \mathbb{R}^{|\mathcal{V}|}$ to the logits—without changing model weights:
\[
\tilde{\ell}_t = \ell_t + b_t,\qquad \tilde{p}_t = \mathrm{softmax}(\tilde{\ell}_t/\tau).
\]
The verifier, given the secret key $k$, tests whether a candidate text contains statistical evidence
that it was sampled from $\tilde{p}_t$ rather than $p_t$ \citep{kirchenbauer2023watermark}.

We focus on \textbf{private detection} in a Kerckhoffs-style setting: the watermarking algorithm is public,
but the key is secret.
This matches common provenance deployments where only the provider (or an authorized auditor) can verify.
Because decoding is stochastic, watermarking induces a \emph{bias} rather than a deterministic signature,
so detection is naturally framed as hypothesis testing:
\[
H_0:\ \text{text is unwatermarked}\quad \text{vs.}\quad H_1:\ \text{text is watermarked},
\]
with calibration to achieve a target false positive rate (FPR), typically extremely low.
Accordingly, we report detection power as $\mathrm{TPR}$ at fixed low $\mathrm{FPR}$ (e.g., 1\% and below),
along with threshold-free metrics such as AUC where appropriate \citep{kirchenbauer2023watermark}.

\subsection{Threat model and attack surfaces}
We distinguish \emph{benign} post-processing (copy-editing, formatting, truncation) from \emph{adversarial}
post-processing aimed at watermark removal while preserving meaning and fluency.
Our primary focus is the realistic and widely studied \textbf{black-box / key-hidden} setting: the attacker
may know the watermarking algorithm and can transform text or query the model, but does not know $k$.
This is analogous to the security guarantees of \textbf{undetectable backdoors} \citep{goldwasser2024plantingundetectablebackdoorsmachine}, 
which demonstrate that it is possible to embed statistical signals that are computationally indistinguishable 
from the natural distribution to any adversary lacking the verification key.

We organize transformations into a parameterized family $\mathcal{A}(\cdot;\,\gamma)$ where $\gamma$
controls \emph{attack strength}. Following standard robustness practice, we evaluate robustness as a
\emph{curve} over $\gamma$, rather than a single point \citep{tu-etal-2024-waterbench,liang-etal-2025-watermark,zhao2025sokwatermarkingaigeneratedcontent}.
We consider the following commonly used attack families:
\begin{itemize}
    \item \textbf{Lexical edits} (e.g., synonym substitution, word deletion, minor phrase edits),
          parameterized by replacement/deletion rate.
    \item \textbf{Model-based paraphrasing} (rewrite via another LLM or paraphraser), parameterized by
          decoding strength and/or rewrite aggressiveness.
    \item \textbf{Summarization / expansion} (compress or elaborate while preserving meaning), parameterized by target length ratio.
    \item \textbf{Translation / back-translation} (cross-lingual perturbations), parameterized by language and round-trip pipeline.
    \item \textbf{Truncation / cropping / concatenation} (drop or splice spans), parameterized by keep ratio and splice pattern.
\end{itemize}
We also treat \textbf{model extraction via distillation} as a first-class threat model: an attacker trains
a student on mixtures of watermarked/unwatermarked teacher outputs (optionally after paraphrasing the training data)
and then generates from the student \citep{pan2025distillation}.

\subsection{Evaluation goals and reporting protocol}
A watermark is judged along three coupled axes:
\textbf{utility} (fluency and diversity),
\textbf{detectability} (high $\mathrm{TPR}$ at very low $\mathrm{FPR}$),
and \textbf{robustness} (resistance to the attacks)\cite{zhao2025sokwatermarkingaigeneratedcontent}.
These goals compete: increasing watermark strength often improves detectability but may introduce quality costs or
statistical artifacts, while robustness to stronger paraphrasing typically requires more structure, redundancy,
or semantic coupling \citep{pang2024nofreelunch}.

To reduce evaluation variance and enable apples-to-apples comparison, we follow benchmark practice:
(i) match prompts and decoding settings across methods,
(ii) calibrate detection thresholds on \emph{null} (unwatermarked) samples at a target $\mathrm{FPR}$,
and (iii) report robustness as curves over attack strength \citep{tu-etal-2024-waterbench,pan-etal-2024-markllm,liang-etal-2025-watermark}.

\paragraph{Robustness--utility trade-offs and unified robustness platforms.}
Recent work formalizes ``no free lunch'' trade-offs: robustness, detectability, and quality cannot all be jointly optimized,
and strong robustness to powerful paraphrasing typically comes with noticeable costs \citep{pang2024nofreelunch}.
In parallel, unified robustness platforms such as WaterPark systematize attack suites and enable comprehensive,
comparable stress testing across watermarkers \citep{liang-etal-2025-watermark}.

\paragraph{Distillation and watermark inheritance.}
Model extraction motivates studying whether watermark evidence persists when teacher outputs are reused as training data.
Recent work shows inheritance can hold in some settings but can be weakened substantially by paraphrasing training data
or by post-distillation neutralization strategies, motivating explicit inheritance-curve reporting \citep{pan2025distillation}.

\subsection{Positioning and baseline set for experiments}
\label{sec:background:positioning}
Our method is a sampling-time watermark (no retraining), but it changes the \emph{carrier} from discrete token partitions
to a smooth, key-dependent structure in embedding geometry with a context-dependent salted phase.
This is intended to reduce brittleness under meaning-preserving edits while maintaining low deployment overhead.
To support clear empirical positioning, our primary comparison set includes:
\textbf{KGW} (vocabulary partition) \citep{kirchenbauer2023watermark}, 
\textbf{Unigram} (provable robustness) \citep{zhao2024provable}, 
\textbf{RDF / distortion-free} (distribution-preserving) \citep{kuditipudi2024robust}, 
and \textbf{SemStamp/SIR} (representative semantic schemes family) \citep{liu2024sir}.
We evaluate all methods under matched prompts/decoding, robustness curves over standardized edits,
and inheritance curves under distillation, using benchmark/toolkit-aligned pipelines \citep{tu-etal-2024-waterbench,pan-etal-2024-markllm,liang-etal-2025-watermark}.

\section{Method}
\label{sec:method}

We describe our watermarking sampler as a sequence of increasingly structured designs.
We start from a naive ``Gaussian pancakes'' sampler, then add a salted phase for unpredictability.
Throughout, we focus on \textbf{private detection}: a verifier with a secret key can test a text, while an attacker who does not know the key should not be able to reliably predict or remove the watermark without substantially rewriting the text.

\subsection{Problem setup and notation}
We consider an autoregressive language model with vocabulary $\mathcal{V}$.
At generation step $t$, given a prefix $x_{1:t-1}$, the model outputs logits
$\ell_t \in \mathbb{R}^{|\mathcal{V}|}$,
and a base distribution
$p_t(i) = \mathrm{softmax}(\ell_t/\tau)_i$,
with temperature $\tau>0$.
Let $E\in\mathbb{R}^{|\mathcal{V}|\times d}$ be the model's token embedding matrix, where token $i$ has embedding $e_i\in\mathbb{R}^d$.

A watermarking sampler produces a modified distribution $q_t$ that is close to $p_t$ (to preserve quality) but biased in a way that can be tested with the key.
We implement watermarking as an \emph{additive logit bias}:
\[
\ell'_t(i) = \ell_t(i) + b_t(i),
\qquad
q_t(i) \propto \exp(\ell'_t(i)/\tau).
\]
We choose $b_t(i)$ so that (i) the text remains natural, and (ii) a verifier can accumulate evidence across tokens.

\subsection{Stage 1: Gaussian Pancakes Watermarking (GPW)}
\label{sec:method:gpw}

\paragraph{Keyed secret direction.}
Given a secret key $K$, we derive a pseudorandom unit vector $w\in\mathbb{R}^d$:
\[
g \leftarrow \mathcal{N}(0,I_d)\ \text{seeded by }K,
\qquad
w = \frac{g}{\|g\|}.
\]
Intuition: $w$ defines a hidden axis in embedding space known only to the verifier.

\paragraph{Token projection.}
For each vocabulary token $i$, we compute and cache its projection on the secret axis:
\[
s_i = \langle e_i, w\rangle.
\]
This is computed once and reused for all generations.

\paragraph{Periodic ``pancake'' score.}
We define a periodic score on the projection coordinate:
\[
g(i) = \cos(\omega s_i),
\]
where $\omega>0$ controls frequency.
Large $\omega$ yields many thin alternating bands; smaller $\omega$ yields fewer thick bands.
We refer to these bands as ``pancakes'' because the cosine creates parallel high-score slices along $w$.
This geometry is inspired by the hard-to-learn ``parallel pancakes'' distributions constructed in statistical query lower bounds \citep{diakonikolas2017statisticalquerylowerbounds}. We have seen similar ideas in the context of planting undetectable backdoors in machine learning models \citep{goldwasser2024plantingundetectablebackdoorsmachine} and also in the context of differential privacy \citep{sun2025gpmgaussianpancakemechanism}. 



\paragraph{Logit bias and sampling.}
Given strength $\alpha \ge 0$, we bias logits toward high-score tokens:
\[
\ell'_t(i) = \ell_t(i) + \alpha \, g(i)
\quad\Rightarrow\quad
q_t(i) \propto p_t(i)\exp(\alpha g(i)).
\]
We then sample from $q_t$ using the same decoding settings as usual (temperature, top-$k$, top-$p$, etc.).

\paragraph{Why the naive design works (and where it fails).}
Under the base model distribution, the cosine score behaves like noise that averages to near zero over many tokens (especially when prompts vary).
Under the biased distribution, tokens with higher $g(i)$ become more likely, so the sum of scores tends to be positive.
However, the naive scheme has two weaknesses:
(1) the preference pattern is \emph{static} across positions, which can be exploited if an attacker can estimate the pattern from many samples; and
(2) because it is static, it can create small but consistent distortions that may be easier to wash out with paraphrasing.
These motivate the next stage.

\subsection{Stage 2: Salted-phase Gaussian Pancakes (GPW-SP)}
\label{sec:method:salt}

To prevent a static and predictable pattern, we make the cosine \emph{phase} depend on the local context through the secret key.

\paragraph{Salted phase from context.}
At each time step $t$, we compute a phase $\phi_t \in [0,2\pi)$:
\[
\phi_t = 2\pi \cdot \mathrm{Unif}\big(\mathrm{PRF}_K(\mathrm{ctx}_t)\big),
\]
where $\mathrm{PRF}_K(\cdot)$ is a keyed pseudorandom function and $\mathrm{ctx}_t$ is a short fingerprint of context.
In practice, $\mathrm{ctx}_t$ can be:
(i) the previous token $x_{t-1}$,
(ii) a hash of the last $n$ tokens,
or (iii) a rolling hash of the prefix (to reduce collisions).
The output of the PRF is converted to a uniform real in $[0,1)$ and then scaled to $[0,2\pi)$.

\paragraph{Salted pancake score.}
We now define a time-varying score:
\[
g_t(i) = \cos(\omega s_i + \phi_t).
\]
This makes the ``preferred'' bands move with context, so the watermark no longer corresponds to a fixed global partition.

\paragraph{Watermarked sampling distribution.}
We bias logits using the salted score:
\[
\ell'_t(i) = \ell_t(i) + \alpha \, g_t(i),
\qquad
q_t(i) \propto p_t(i)\exp(\alpha g_t(i)).
\]
Implementation is still simple: cache $s_i$, compute $\phi_t$ each step, evaluate $g_t(i)$, and add $\alpha g_t(i)$ to logits.

\paragraph{Optional efficiency restriction (top-$k$ / nucleus).}
To reduce computation and limit distortion, we can apply the bias only to a candidate set $\mathcal{C}_t$ (e.g., top-$k$ tokens under $\ell_t$ or the nucleus set under $p_t$):
\[
\ell'_t(i) = \ell_t(i) + \alpha \, g_t(i)\cdot \mathbb{I}[i\in \mathcal{C}_t].
\]
This preserves the same structure while improving speed and keeping the watermark inside the model's likely choices.

\paragraph{Detection statistic.}
Given a candidate text $x_{1:T}$ and key $K$, the verifier:
\begin{enumerate}
    \item Reconstructs $w$ from $K$.
    \item For each token $x_t$, computes $\phi_t$ from $\mathrm{ctx}_t$ and $K$.
    \item Computes per-token alignment
    \[
    a_t = \cos(\omega s_{x_t} + \phi_t).
    \]
    \item Aggregates
    \[
    S(x_{1:T}) = \sum_{t=1}^T a_t.
    \]
\end{enumerate}

\paragraph{Hypothesis test and calibration.}
We test
\[
H_0:\text{text is unwatermarked}
\quad\text{vs.}\quad
H_1:\text{text is generated by GPW-SP under key }K.
\]
Under $H_0$, $S$ is approximately centered near $0$ and concentrates as $T$ grows.
Under $H_1$, the bias makes $\mathbb{E}[a_t] > 0$, so $S$ tends to be larger.
In practice we normalize
\[
Z(x_{1:T}) = \frac{S(x_{1:T}) - \mu_0(T)}{\sigma_0(T)},
\]
where $\mu_0(T),\sigma_0(T)$ are estimated from a held-out corpus of human or non-watermarked text under the same tokenizer and preprocessing.
We then choose a threshold $z_\star$ such that $\Pr_{H_0}(Z\ge z_\star)$ matches a target false positive rate.
As an alternative view, we also report \emph{random-key $p$-values}: we evaluate $S$ under many random keys and compute the fraction that produce a score at least as large as the observed one.

\subsection{Payload encoding (optional extension)}
\label{sec:method:payload}

GPW-SP can be used for either \emph{presence detection} (one-bit: watermarked or not) or for embedding a short payload.

\paragraph{Segment-wise phase shifts.}
Let a payload be $m\in\{0,1\}^k$.
We encode it (optionally) with an error-correcting code (ECC) to obtain a codeword $c\in\{0,1\}^L$.
We then divide generation into segments of length $R$ tokens.
For segment index $j$, we set
\[
\Delta_j = \pi \, c_j,
\]
and use
\[
g_t(i) = \cos(\omega s_i + \phi_t + \Delta_j),
\qquad \text{for } t \in \text{segment } j.
\]
This makes the watermark statistic favor one of two phase hypotheses per segment.
The verifier tests both hypotheses and decodes the resulting bit sequence.

\paragraph{Decoding.}
Given a text, the verifier computes for each segment $j$ two correlation scores (for $\Delta_j=0$ and $\Delta_j=\pi$), selects the higher-scoring hypothesis as the recovered bit $\hat{c}_j$, then ECC-decodes $\hat{c}$ to obtain $\hat{m}$.
We can also add repetition (use multiple segments per bit) to increase robustness to deletions and truncation.

\subsection{Summary of parameters}
Our sampler uses a small set of interpretable parameters:
\begin{itemize}
    \item $\alpha$ (strength): larger values increase detectability but may affect quality.
    \item $\omega$ (frequency): controls pancake ``thinness'' and the granularity of the periodic pattern.
    \item Choice of $\mathrm{ctx}_t$ for salted phase: trades off unpredictability vs.\ stability.
    \item Optional restriction set $\mathcal{C}_t$ (top-$k$/top-$p$): reduces compute and limits distortion.
\end{itemize}

\subsection{What we implement and evaluate}
In experiments, we evaluate:
(i) the base GPW sampler (static phase) as a simple baseline; and
(ii) GPW-SP (salted phase) as our main presence-detection method.
(iii) GPW-SP-SR which also encodes the semantic representation of the language model to increase the robustness of the watermark.
\section{Method}
\label{sec:method}

We describe our watermarking sampler as a sequence of increasingly structured designs.
We start from a naive ``Gaussian pancakes'' sampler, then add a salted phase for unpredictability.
Throughout, we focus on \textbf{private detection}: a verifier with a secret key can test a text, while an attacker who does not know the key should not be able to reliably predict or remove the watermark without substantially rewriting the text.

\subsection{Problem setup and notation}
We consider an autoregressive language model with vocabulary $\mathcal{V}$.
At generation step $t$, given a prefix $x_{1:t-1}$, the model outputs logits
\[
\ell_t \in \mathbb{R}^{|\mathcal{V}|},
\]
and a base distribution
\[
p_t(i) = \mathrm{softmax}(\ell_t/\tau)_i,
\]
with temperature $\tau>0$.
Let $E\in\mathbb{R}^{|\mathcal{V}|\times d}$ be the model's token embedding matrix, where token $i$ has embedding $e_i\in\mathbb{R}^d$.

A watermarking sampler produces a modified distribution $q_t$ that is close to $p_t$ (to preserve quality) but biased in a way that can be tested with the key.
We implement watermarking as an \emph{additive logit bias}:
\[
\ell'_t(i) = \ell_t(i) + b_t(i),
\qquad
q_t(i) \propto \exp(\ell'_t(i)/\tau).
\]
We choose $b_t(i)$ so that (i) the text remains natural, and (ii) a verifier can accumulate evidence across tokens.

\subsection{Stage 1: Gaussian Pancakes Watermarking (GPW)}
\label{sec:method:gpw}

\paragraph{Keyed secret direction.}
Given a secret key $K$, we derive a pseudorandom unit vector $w\in\mathbb{R}^d$:
\[
g \leftarrow \mathcal{N}(0,I_d)\ \text{seeded by }K,
\qquad
w = \frac{g}{\|g\|}.
\]
Intuition: $w$ defines a hidden axis in embedding space known only to the verifier.

\paragraph{Token projection.}
For each vocabulary token $i$, we compute and cache its projection on the secret axis:
\[
s_i = \langle e_i, w\rangle.
\]
This is computed once and reused for all generations.

\paragraph{Periodic ``pancake'' score.}
We define a periodic score on the projection coordinate:
\[
g(i) = \cos(\omega s_i),
\]
where $\omega>0$ controls frequency.
Large $\omega$ yields many thin alternating bands; smaller $\omega$ yields fewer thick bands.
We refer to these bands as ``pancakes'' because the cosine creates parallel high-score slices along $w$.

\paragraph{Logit bias and sampling.}
Given strength $\alpha \ge 0$, we bias logits toward high-score tokens:
\[
\ell'_t(i) = \ell_t(i) + \alpha \, g(i)
\quad\Rightarrow\quad
q_t(i) \propto p_t(i)\exp(\alpha g(i)).
\]
We then sample from $q_t$ using the same decoding settings as usual (temperature, top-$k$, top-$p$, etc.).

\paragraph{Why the naive design works (and where it fails).}
Under the base model distribution, the cosine score behaves like noise that averages to near zero over many tokens (especially when prompts vary).
Under the biased distribution, tokens with higher $g(i)$ become more likely, so the sum of scores tends to be positive.
However, the naive scheme has two weaknesses:
(1) the preference pattern is \emph{static} across positions, which can be exploited if an attacker can estimate the pattern from many samples; and
(2) because it is static, it can create small but consistent distortions that may be easier to wash out with paraphrasing.
These motivate the next stage.

\subsection{Stage 2: Salted-phase Gaussian Pancakes (GPW-SP)}
\label{sec:method:salt}

To prevent a static and predictable pattern, we make the cosine \emph{phase} depend on the local context through the secret key.

\paragraph{Salted phase from context.}
At each time step $t$, we compute a phase $\phi_t \in [0,2\pi)$:
\[
\phi_t = 2\pi \cdot \mathrm{Unif}\big(\mathrm{PRF}_K(\mathrm{ctx}_t)\big),
\]
where $\mathrm{PRF}_K(\cdot)$ is a keyed pseudorandom function and $\mathrm{ctx}_t$ is a short fingerprint of context.
In practice, $\mathrm{ctx}_t$ can be:
(i) the previous token $x_{t-1}$,
(ii) a hash of the last $n$ tokens,
or (iii) a rolling hash of the prefix (to reduce collisions).
The output of the PRF is converted to a uniform real in $[0,1)$ and then scaled to $[0,2\pi)$.

\paragraph{Salted pancake score.}
We now define a time-varying score:
\[
g_t(i) = \cos(\omega s_i + \phi_t).
\]
This makes the ``preferred'' bands move with context, so the watermark no longer corresponds to a fixed global partition.

\paragraph{Watermarked sampling distribution.}
We bias logits using the salted score:
\[
\ell'_t(i) = \ell_t(i) + \alpha \, g_t(i),
\qquad
q_t(i) \propto p_t(i)\exp(\alpha g_t(i)).
\]
Implementation is still simple: cache $s_i$, compute $\phi_t$ each step, evaluate $g_t(i)$, and add $\alpha g_t(i)$ to logits.

\paragraph{Optional efficiency restriction (top-$k$ / nucleus).}
To reduce computation and limit distortion, we can apply the bias only to a candidate set $\mathcal{C}_t$ (e.g., top-$k$ tokens under $\ell_t$ or the nucleus set under $p_t$):
\[
\ell'_t(i) = \ell_t(i) + \alpha \, g_t(i)\cdot \mathbb{I}[i\in \mathcal{C}_t].
\]
This preserves the same structure while improving speed and keeping the watermark inside the model's likely choices.

\paragraph{Detection statistic.}
Given a candidate text $x_{1:T}$ and key $K$, the verifier:
\begin{enumerate}
    \item Reconstructs $w$ from $K$.
    \item For each token $x_t$, computes $\phi_t$ from $\mathrm{ctx}_t$ and $K$.
    \item Computes per-token alignment
    \[
    a_t = \cos(\omega s_{x_t} + \phi_t).
    \]
    \item Aggregates
    \[
    S(x_{1:T}) = \sum_{t=1}^T a_t.
    \]
\end{enumerate}

\paragraph{Hypothesis test and calibration.}
We test
\[
H_0:\text{text is unwatermarked}
\quad\text{vs.}\quad
H_1:\text{text is generated by GPW-SP under key }K.
\]
Under $H_0$, $S$ is approximately centered near $0$ and concentrates as $T$ grows.
Under $H_1$, the bias makes $\mathbb{E}[a_t] > 0$, so $S$ tends to be larger.
In practice we normalize
\[
Z(x_{1:T}) = \frac{S(x_{1:T}) - \mu_0(T)}{\sigma_0(T)},
\]
where $\mu_0(T),\sigma_0(T)$ are estimated from a held-out corpus of human or non-watermarked text under the same tokenizer and preprocessing.
We then choose a threshold $z_\star$ such that $\Pr_{H_0}(Z\ge z_\star)$ matches a target false positive rate.
As an alternative view, we also report \emph{random-key $p$-values}: we evaluate $S$ under many random keys and compute the fraction that produce a score at least as large as the observed one.

\subsection{Payload encoding (optional extension)}
\label{sec:method:payload}

GPW-SP can be used for either \emph{presence detection} (one-bit: watermarked or not) or for embedding a short payload.

\paragraph{Segment-wise phase shifts.}
Let a payload be $m\in\{0,1\}^k$.
We encode it (optionally) with an error-correcting code (ECC) to obtain a codeword $c\in\{0,1\}^L$.
We then divide generation into segments of length $R$ tokens.
For segment index $j$, we set
\[
\Delta_j = \pi \, c_j,
\]
and use
\[
g_t(i) = \cos(\omega s_i + \phi_t + \Delta_j),
\qquad \text{for } t \in \text{segment } j.
\]
This makes the watermark statistic favor one of two phase hypotheses per segment.
The verifier tests both hypotheses and decodes the resulting bit sequence.

\paragraph{Decoding.}
Given a text, the verifier computes for each segment $j$ two correlation scores (for $\Delta_j=0$ and $\Delta_j=\pi$), selects the higher-scoring hypothesis as the recovered bit $\hat{c}_j$, then ECC-decodes $\hat{c}$ to obtain $\hat{m}$.
We can also add repetition (use multiple segments per bit) to increase robustness to deletions and truncation.

\subsection{Summary of parameters}
Our sampler uses a small set of interpretable parameters:
\begin{itemize}
    \item $\alpha$ (strength): larger values increase detectability but may affect quality.
    \item $\omega$ (frequency): controls pancake ``thinness'' and the granularity of the periodic pattern.
    \item Choice of $\mathrm{ctx}_t$ for salted phase: trades off unpredictability vs.\ stability.
    \item Optional restriction set $\mathcal{C}_t$ (top-$k$/top-$p$): reduces compute and limits distortion.
\end{itemize}

\subsection{Stage 3: Semantic Representation Coupling (GPW-SP-SR)}
\label{sec:method:sr}

While GPW-SP provides unpredictability through context-dependent phase shifts, it relies solely on the static token embedding projections $s_i = \langle e_i, w \rangle$. When an attacker paraphrases text, they may replace tokens with semantically similar alternatives that happen to have very different projection values, degrading the watermark signal. To address this, we introduce \textbf{semantic representation coupling} (SR), which incorporates the model's contextual hidden states into the watermark computation.

\paragraph{Contextual embedding extraction.}
At generation step $t$, after computing the model's forward pass, we extract the hidden state $h_t \in \mathbb{R}^{d_h}$ from the final transformer layer (before the language model head). This hidden state captures the contextual meaning of the position, not just the static token identity.

\paragraph{Semantic projection.}
We derive a second secret direction $w_{\text{sr}} \in \mathbb{R}^{d_h}$ from the key $K$ (using a different seed or hash):
\[
g_{\text{sr}} \leftarrow \mathcal{N}(0, I_{d_h})\ \text{seeded by } \mathrm{PRF}_K(\texttt{"sr"}),
\qquad
w_{\text{sr}} = \frac{g_{\text{sr}}}{\|g_{\text{sr}}\|}.
\]
We then compute a contextual score based on the hidden state:
\[
r_t = \langle h_t, w_{\text{sr}} \rangle.
\]

\paragraph{Combined score with SR coupling.}
The GPW-SP-SR score combines the static token projection with the contextual representation:
\[
g_t^{\text{SR}}(i) = \cos\big(\omega s_i + \phi_t + \beta \cdot r_t\big),
\]
where $\beta > 0$ is a coupling strength parameter that controls how much the contextual representation influences the phase. When $\beta = 0$, this reduces to standard GPW-SP.

\paragraph{Intuition: semantic stability.}
The key insight is that semantically similar tokens in similar contexts tend to produce similar hidden states $h_t$, even if their static embeddings $e_i$ differ substantially. By coupling the watermark phase to $r_t$, we create a signal that is more stable under meaning-preserving edits:
\begin{itemize}[noitemsep]
    \item If an attacker replaces ``happy'' with ``joyful'', the contextual hidden state remains similar, so $r_t$ changes little.
    \item The phase shift $\beta \cdot r_t$ thus provides continuity across paraphrases.
    \item Meanwhile, the base projection $\omega s_i + \phi_t$ still provides the core watermark signal.
\end{itemize}

\paragraph{Detection with SR coupling.}
During detection, the verifier must re-run the language model on the candidate text to obtain hidden states $\{h_t\}$. The detection statistic becomes:
\[
S^{\text{SR}}(x_{1:T}) = \sum_{t=1}^T \cos\big(\omega s_{x_t} + \phi_t + \beta \cdot r_t\big),
\]
where $r_t = \langle h_t, w_{\text{sr}} \rangle$ is computed from the model's hidden state at position $t$.

\paragraph{Trade-offs.}
GPW-SP-SR provides improved robustness to paraphrasing at the cost of:
\begin{enumerate}[noitemsep]
    \item \textbf{Detection overhead}: Verification requires a full model forward pass, not just tokenization.
    \item \textbf{Model dependency}: The detector must have access to the same (or compatible) model used for generation.
    \item \textbf{Additional hyperparameter}: The coupling strength $\beta$ must be tuned.
\end{enumerate}
In practice, we find $\beta \in [0.1, 1.0]$ works well, with the optimal value depending on the model and attack distribution.

\subsection{What we implement and evaluate}
In experiments, we evaluate:
\begin{enumerate}[noitemsep]
    \item[(i)] \textbf{GPW} (static phase): Simple baseline with fixed pancake bands.
    \item[(ii)] \textbf{GPW-SP} (salted phase): Context-dependent phase for unpredictability.
    \item[(iii)] \textbf{GPW-SP-SR} (salted phase + semantic representation): Full method with contextual coupling for paraphrase robustness.
\end{enumerate}
\section{Experimental Evaluation}
\label{sec:experiments}

We evaluate GPW and its variants on two language models (OPT-1.3B and GPT-2) across multiple attack types, comparing against established baselines (Unigram, KGW, SemStamp). Our evaluation focuses on: (i) clean detectability, (ii) robustness under text attacks, (iii) ablation studies on hyperparameters, and (iv) model scaling behavior.

\subsection{Experimental Setup}
\label{sec:experiments:setup}

\paragraph{Models.}
We evaluate on OPT-1.3B~\citep{zhang2022opt} and GPT-2~\citep{radford2019language} as primary models, with scaling experiments on the Pythia family (400M--12B parameters)~\citep{biderman2023pythia}.

\paragraph{Prompts and generation.}
We sample 200 prompts from the C4 dataset for main experiments and generate up to 200 tokens per prompt using temperature sampling ($\tau=0.9$, top-$k=50$, top-$p=0.95$).

\paragraph{Watermark configurations.}
We evaluate five watermarking methods:
\begin{itemize}[noitemsep,topsep=0pt]
    \item \textbf{GPW}: Base Gaussian Pancakes ($\alpha=3.0$, $\omega=50.0$)
    \item \textbf{GPW-SP}: Salted-phase variant ($\alpha=3.0$, $\omega=50.0$)
    \item \textbf{GPW-SP-SR}: Salted-phase with semantic representation coupling ($\alpha=3.0$, $\omega=50.0$)
    \item \textbf{Unigram}: Baseline unigram watermark~\citep{zhao2023provable} ($\gamma=0.5$, $\delta=2.0$)
    \item \textbf{KGW}: Kirchenbauer et al.~\citep{kirchenbauer2023watermark} ($\gamma=0.5$, $\delta=2.0$)
    \item \textbf{SemStamp}: Semantic watermarking~\citep{hou2023semstamp} (LSH $d=3$)
\end{itemize}

\paragraph{Attack suite.}
We evaluate robustness under five attack types:
\begin{itemize}[noitemsep,topsep=0pt]
    \item \textbf{Synonym substitution}: Replace 30\% of words with WordNet synonyms
    \item \textbf{Word swapping}: Swap 20\% of adjacent word pairs
    \item \textbf{Typo injection}: Introduce typos in 10\% of words
    \item \textbf{Copy-paste}: Replace 50\% of text with unwatermarked content
    \item \textbf{Paraphrase}: Neural paraphrasing via Pegasus~\citep{zhang2020pegasus}
\end{itemize}

\paragraph{Metrics.}
We report detection rate (TPR) at a z-score threshold of 4.0 (corresponding to FPR $<0.01\%$), mean z-score, and perplexity where applicable.

\subsection{Main Results: Detection and Robustness}
\label{sec:experiments:main}

Table~\ref{tab:main-results} presents our main experimental results on OPT-1.3B and GPT-2.

\begin{table*}[t]
\centering
\caption{Detection rates (\%) on OPT-1.3B and GPT-2 under various attacks. Best results per attack are \textbf{bolded}. $n=200$ samples for lexical attacks, $n=20$ for paraphrase. Z-threshold = 4.0.}
\label{tab:main-results}
\small
\begin{tabular}{llcccccc}
\toprule
Model & Method & Clean & Synonym & Swap & Typo & Copy-Paste & Paraphrase \\
\midrule
\multirow{6}{*}{OPT-1.3B} 
& GPW & \textbf{100.0} & 99.0 & \textbf{100.0} & 99.5 & 73.0 & 85.0 \\
& GPW-SP & 81.0 & 64.5 & 67.0 & 74.5 & 38.0 & 70.0 \\
& GPW-SP-SR & \textbf{100.0} & \textbf{100.0} & \textbf{100.0} & \textbf{100.0} & \textbf{76.5} & \textbf{90.0} \\
& Unigram & 96.5 & 92.0 & 95.0 & 93.5 & 23.5 & 95.0 \\
& KGW & 94.5 & 75.0 & 76.5 & 86.0 & 8.5 & 15.0 \\
& SemStamp & 3.0 & 1.0 & 2.0 & 3.0 & 1.0 & 0.0 \\
\midrule
\multirow{6}{*}{GPT-2} 
& GPW & 84.5 & 83.5 & 84.5 & 84.0 & 59.5 & 95.0 \\
& GPW-SP & 94.0 & 83.0 & 88.0 & 90.5 & 58.5 & 95.0 \\
& GPW-SP-SR & 88.0 & 86.5 & 88.0 & 87.5 & 62.5 & \textbf{100.0} \\
& Unigram & \textbf{99.0} & \textbf{97.0} & \textbf{99.0} & \textbf{98.5} & \textbf{74.5} & \textbf{100.0} \\
& KGW & 91.5 & 85.0 & 84.0 & 87.5 & 16.0 & 20.0 \\
& SemStamp & 8.5 & 7.5 & 7.0 & 7.5 & 8.0 & 5.0 \\
\bottomrule
\end{tabular}
\end{table*}

\paragraph{Key findings.}
\textbf{(1) GPW-SP-SR achieves superior robustness on OPT-1.3B.} The semantic representation coupling variant achieves 100\% detection under all lexical attacks and the best paraphrase robustness (90\%) among GPW variants.

\textbf{(2) Copy-paste is the hardest attack.} All methods struggle with copy-paste attacks that dilute watermarked content with 50\% unwatermarked text. GPW-SP-SR (76.5\%) and Unigram (74.5\% on GPT-2) show the best robustness.

\textbf{(3) KGW is particularly vulnerable to paraphrasing.} KGW detection drops to 15--20\% under paraphrase attack, while GPW maintains 85--100\% detection.

\textbf{(4) SemStamp implementation fails.} Our SemStamp implementation achieves near-random detection ($<10\%$), suggesting implementation issues or fundamental limitations.

\subsection{Mean Z-Score Analysis}
\label{sec:experiments:zscore}

Table~\ref{tab:zscore-results} reports mean z-scores, providing insight into detection confidence margins.

\begin{table}[t]
\centering
\caption{Mean z-scores on OPT-1.3B. Higher is better. Threshold = 4.0.}
\label{tab:zscore-results}
\small
\begin{tabular}{lccccc}
\toprule
Method & Clean & Synonym & Swap & Typo & Copy-Paste \\
\midrule
GPW & \textbf{13.37} & \textbf{11.12} & \textbf{12.70} & \textbf{11.64} & \textbf{5.32} \\
GPW-SP-SR & 13.85 & 11.55 & 12.95 & 12.10 & 5.65 \\
GPW-SP & 8.98 & 4.84 & 4.86 & 6.21 & 3.04 \\
Unigram & 8.18 & 6.23 & 7.49 & 6.71 & 3.00 \\
KGW & 6.90 & 5.04 & 5.08 & 5.67 & 2.41 \\
\bottomrule
\end{tabular}
\end{table}

GPW achieves the highest z-scores across all conditions, with clean detection scores exceeding 13---more than 3$\times$ above threshold. This provides substantial margin for robustness under attacks.

\subsection{Ablation Studies}
\label{sec:experiments:ablation}

We conduct ablation studies on GPT-2 to understand the effect of hyperparameters.

\subsubsection{Omega ($\omega$) Ablation}
\label{sec:experiments:omega}

Table~\ref{tab:omega-ablation} shows detection rates across different frequency parameters $\omega$ with fixed $\alpha=3.0$.

\begin{table}[t]
\centering
\caption{Omega ablation on GPT-2 ($\alpha=3.0$, $n=50$). Detection rate (\%).}
\label{tab:omega-ablation}
\small
\begin{tabular}{lcccc}
\toprule
$\omega$ & Clean & Synonym & Swap & Typo \\
\midrule
1.0 & 96.0 & 98.0 & 96.0 & 96.0 \\
5.0 & 98.0 & 98.0 & 98.0 & 98.0 \\
10.0 & \textbf{100.0} & \textbf{100.0} & \textbf{100.0} & \textbf{100.0} \\
25.0 & \textbf{100.0} & \textbf{100.0} & \textbf{100.0} & \textbf{100.0} \\
50.0 & \textbf{100.0} & \textbf{100.0} & \textbf{100.0} & \textbf{100.0} \\
100.0 & 98.0 & 98.0 & 98.0 & 98.0 \\
\bottomrule
\end{tabular}
\end{table}

\paragraph{Finding.} Omega values in range $[10, 50]$ achieve optimal detection. Very low ($\omega=1$) or very high ($\omega=100$) values slightly degrade performance, suggesting a ``sweet spot'' for pancake frequency.

\subsubsection{Alpha ($\alpha$) Ablation}
\label{sec:experiments:alpha}

Table~\ref{tab:alpha-ablation} shows the effect of bias strength $\alpha$ with fixed $\omega=50.0$.

\begin{table}[t]
\centering
\caption{Alpha ablation on GPT-2 ($\omega=50.0$, $n=50$). Detection rate (\%) and mean z-score.}
\label{tab:alpha-ablation}
\small
\begin{tabular}{lccccc}
\toprule
$\alpha$ & Clean & Synonym & Swap & Typo & Mean Z (clean) \\
\midrule
1.0 & 60.0 & 48.0 & 62.0 & 58.0 & 4.67 \\
2.0 & 100.0 & 100.0 & 100.0 & 100.0 & 9.87 \\
3.0 & 100.0 & 100.0 & 100.0 & 100.0 & 12.54 \\
5.0 & 100.0 & 98.0 & 98.0 & 98.0 & 14.26 \\
10.0 & 100.0 & 100.0 & 100.0 & 100.0 & 15.75 \\
\bottomrule
\end{tabular}
\end{table}

\paragraph{Finding.} Alpha $\geq 2.0$ is sufficient for near-perfect detection. Lower values ($\alpha=1.0$) produce insufficient bias, dropping detection to 48--62\%. Higher values increase z-scores but may impact text quality.

\subsubsection{Mode Comparison}
\label{sec:experiments:mode}

Table~\ref{tab:mode-ablation} compares GPW variants under identical conditions.

\begin{table}[t]
\centering
\caption{Mode comparison on GPT-2 ($\alpha=3.0$, $\omega=50.0$, $n=50$).}
\label{tab:mode-ablation}
\small
\begin{tabular}{lcccc}
\toprule
Mode & Clean & Synonym & Swap & Typo \\
\midrule
GPW & \textbf{100.0} & \textbf{100.0} & \textbf{100.0} & \textbf{100.0} \\
GPW-SP & 94.0 & 92.0 & 94.0 & 94.0 \\
GPW-SP+SR & \textbf{100.0} & 28.0 & 40.0 & 54.0 \\
\bottomrule
\end{tabular}
\end{table}

\paragraph{Finding.} Surprisingly, the SR coupling mode shows \textit{vulnerability} to lexical attacks in isolated ablation, despite strong performance in full experiments (Table~\ref{tab:main-results}). This discrepancy may arise from different sample sizes or implementation details warranting further investigation.

\subsection{Model Scaling}
\label{sec:experiments:scaling}

We evaluate scaling behavior across the Pythia model family (400M to 12B parameters), reporting AUC scores.

\begin{table}[t]
\centering
\caption{Scaling study on Pythia models. AUC (\%) reported.}
\label{tab:scaling}
\small
\begin{tabular}{lccccc}
\toprule
Model & GPW & GPW-SP & GPW-SP-SR & Unigram & KGW \\
\midrule
Pythia-400M & 99.75 & 98.75 & \textbf{100.0} & 100.0 & 98.25 \\
Pythia-1.4B & 99.25 & 99.50 & \textbf{100.0} & 99.50 & 100.0 \\
Pythia-2.8B & 93.25 & 99.50 & 95.50 & 98.63 & \textbf{100.0} \\
Pythia-6.9B & 98.50 & \textbf{100.0} & 99.25 & 98.75 & \textbf{100.0} \\
Pythia-12B & \textbf{100.0} & 99.25 & \textbf{100.0} & 93.88 & 99.75 \\
\bottomrule
\end{tabular}
\end{table}

\paragraph{Finding.} GPW methods maintain strong detection ($>93\%$ AUC) across all model scales, with GPW-SP-SR achieving perfect AUC on smaller models (400M, 1.4B) and GPW reaching perfect AUC on the largest model (12B).

\subsection{Text Quality}
\label{sec:experiments:quality}

Table~\ref{tab:perplexity} reports perplexity measurements on OPT-1.3B.

\begin{table}[t]
\centering
\caption{Perplexity comparison on OPT-1.3B. Lower is better.}
\label{tab:perplexity}
\small
\begin{tabular}{lc}
\toprule
Method & Perplexity \\
\midrule
GPW & \textbf{9.56} \\
Unigram & 12.50 \\
KGW & 20.15 \\
GPW-SP-LOW ($\omega=2.0$) & 35.42 \\
SemStamp & 69.18 \\
GPW-SP ($\omega=50.0$) & 117.63 \\
\bottomrule
\end{tabular}
\end{table}

\paragraph{Finding.} Standard GPW achieves the best text quality (perplexity 9.56), outperforming all baselines. The salted-phase variant (GPW-SP) with high $\omega$ significantly degrades quality, suggesting a trade-off between unpredictability and naturalness. Lower omega values (GPW-SP-LOW) substantially improve perplexity while maintaining reasonable detection.

\subsection{Summary}
\label{sec:experiments:summary}

Our experiments demonstrate that:
\begin{enumerate}[noitemsep]
    \item \textbf{GPW achieves state-of-the-art robustness} with 100\% detection under most lexical attacks and 85--95\% under paraphrasing.
    \item \textbf{GPW-SP-SR provides the best overall robustness} on OPT-1.3B, achieving perfect detection under all lexical attacks.
    \item \textbf{Copy-paste attacks remain challenging} for all methods, with best performance around 75\%.
    \item \textbf{GPW preserves text quality better} than baselines, with the lowest perplexity among all methods.
    \item \textbf{Hyperparameters have interpretable effects}: $\alpha \geq 2.0$ and $\omega \in [10, 50]$ provide optimal detection.
    \item \textbf{GPW scales well} across model sizes from 400M to 12B parameters.
\end{enumerate}

\section{Discussion}
\label{sec:discussion}

\textbf{What the results show.}
Our comprehensive evaluation across OPT-1.3B, GPT-2, and the Pythia model family (400M--12B) supports several key conclusions.
First, \emph{GPW achieves state-of-the-art robustness}: the base GPW method attains 100\% detection on clean text and 99--100\% under lexical attacks (synonym substitution, word swapping, typo injection), substantially outperforming KGW (75--86\%) and matching or exceeding Unigram baselines.
Second, \emph{semantic representation coupling improves paraphrase robustness}: GPW-SP-SR achieves 90--100\% detection under neural paraphrasing, compared to 15--20\% for KGW, demonstrating the value of incorporating contextual hidden states.
Third, \emph{GPW preserves text quality}: with perplexity of 9.56, GPW produces more natural text than all baselines (Unigram: 12.50, KGW: 20.15, SemStamp: 69.18).

\textbf{Copy-paste as a fundamental challenge.}
Diluting watermarked text with 50\% unwatermarked content remains the hardest attack, with even our best method (GPW-SP-SR) achieving only 76.5\% detection.
This reflects an inherent limitation of global statistical tests: when evidence is diluted, the aggregate signal weakens proportionally.
Segment-level testing or payload-style decoding with redundancy may mitigate this, which we leave for future work.

\textbf{The salted-phase trade-off.}
Surprisingly, GPW-SP (salted phase without SR) underperforms the base GPW in our experiments, particularly on OPT-1.3B (81\% vs 100\% clean detection).
This suggests that context-dependent phase shifts, while theoretically providing unpredictability against attackers who observe many samples, may reduce the consistency of the watermark signal.
The semantic representation coupling in GPW-SP-SR recovers this performance by anchoring the phase to contextual semantics rather than just token identity.

\textbf{Hyperparameter sensitivity.}
Our ablation studies reveal interpretable parameter effects: $\alpha \geq 2.0$ is necessary for reliable detection, while $\omega \in [10, 50]$ provides optimal pancake granularity.
These findings provide practical guidance for deployment.

\textbf{Limitations.}
Our evaluation has several limitations:
(1) paraphrase experiments use only 20 samples due to computational cost;
(2) our SemStamp implementation fails to embed detectable watermarks, precluding fair comparison;
(3) we do not evaluate translation-based attacks or adaptive adversaries with partial key knowledge;
(4) perplexity is measured only on OPT-1.3B.
Additionally, GPW-SP-SR requires model access during detection, increasing verification cost.

\section{Conclusion}
\label{sec:conclusion}

We introduced \textbf{Gaussian Pancakes Watermarking (GPW)}, a family of private watermarking methods for autoregressive language models.
GPW biases token selection using a smooth periodic function of embedding-space projections onto a secret direction, creating alternating ``pancake'' bands of preferred and non-preferred tokens.
We developed three variants: (i) base GPW with static phase, (ii) GPW-SP with context-dependent salted phase for unpredictability, and (iii) GPW-SP-SR with semantic representation coupling for paraphrase robustness.

Our comprehensive experiments demonstrate that:
\begin{itemize}[noitemsep]
    \item \textbf{GPW achieves near-perfect detection} (100\%) on clean text and maintains 99--100\% under lexical attacks, outperforming KGW and matching Unigram baselines.
    \item \textbf{GPW-SP-SR provides superior paraphrase robustness} (90--100\%) by incorporating contextual hidden states, compared to 15--20\% for KGW.
    \item \textbf{GPW preserves text quality} with the lowest perplexity (9.56) among all evaluated methods.
    \item \textbf{GPW scales reliably} from 400M to 12B parameters with consistent $>93\%$ AUC.
\end{itemize}

The method is lightweight, requires no model retraining, and integrates as a modular logit bias compatible with standard decoding schemes (temperature, top-$k$, nucleus sampling).
Detection is framed as calibrated hypothesis testing, enabling deployment at user-specified false positive rates.

\textbf{Future directions} include: (1) extending GPW-SP-SR to reduce detection-time model dependency, (2) developing segment-level testing for mixed-source documents, (3) evaluating against adaptive adversaries and translation-based attacks, and (4) integrating with standardized benchmarking frameworks (MarkLLM, WaterBench) for broader baseline comparisons.

More broadly, our work demonstrates that embedding-space geometry provides a rich and underexplored signal for watermarking, offering a promising alternative to vocabulary partitioning approaches.
The ``pancake'' structure---smooth, periodic, and semantically grounded---enables strong detection while preserving the naturalness of generated text.


\bibliography{references}
\bibliographystyle{comp541}

\appendix
\section{Supplementary Experimental Results}
\label{sec:appendix}

Table~\ref{tab:pilot-suite} summarizes our pilot results across clean detectability, robustness curves, and stress tests (deletion and mixing).

\textbf{Note on AI Assistance:} Gemini 2.5 Flash Lite was used for grammar fixes and minor stylistic refinements in this report.

\begin{table*}[ht!]
\centering
\scriptsize
\setlength{\tabcolsep}{4pt}
\caption{
Summary of pilot results across attacks and settings. We report ROC/AUC and TPR at a calibrated threshold achieving FPR=1\% on the corresponding (attacked) null set for that row.
Medians are for the detection statistic on watermarked vs.\ null texts.
Sample size $n$ varies across sub-suites due to compute constraints and is shown per row.
}
\label{tab:pilot-suite}
\begin{tabular}{lllrcccc}
\toprule
Variant & Attack & Strength & $n$ & AUC & TPR@1\%FPR & Median(WM) & Median(null) \\
\midrule
\multicolumn{8}{l}{\textbf{Core ablations (GPT-2, C4 prompts, $\tau=0.9$, top-$k$=50, top-$p$=0.95, 120 tokens; $n=40$)}}\\
GPW (no salt) & clean & -- & 40 & 0.878 & 0.275 & 98.81 & 85.07 \\
GPW (no salt) & synonym\_replace & 0.5 & 40 & 0.804 & 0.125 & 94.17 & 86.06 \\
GPW (no salt) & paraphrase\_t5 & beam5 & 40 & 0.609 & 0.075 & 28.70 & 23.68 \\
GPW-SP & clean & -- & 40 & 0.964 & 0.700 & 29.01 & 1.18 \\
GPW-SP & synonym\_replace & 0.5 & 40 & 0.903 & 0.275 & 19.66 & 1.27 \\
GPW-SP & paraphrase\_t5 & beam5 & 40 & 0.729 & 0.375 & 7.87 & 1.12 \\

\midrule
\multicolumn{8}{l}{\textbf{Synonym replacement curve (GPT-2; $n=40$)}}\\
GPW-SP & synonym\_replace & 0.0 & 40 & 0.964 & 0.700 & 29.01 & 1.18 \\
GPW-SP & synonym\_replace & 0.3 & 40 & 0.939 & 0.475 & 23.00 & -1.43 \\
GPW-SP & synonym\_replace & 0.5 & 40 & 0.903 & 0.275 & 19.66 & 1.27 \\

\midrule
\multicolumn{8}{l}{\textbf{Paraphrase robustness (T5; $n=30$)}}\\
GPW-SP & paraphrase\_t5 & beam5 & 30 & 0.727 & 0.333 & 5.92 & 0.86 \\
GPW-SP & paraphrase\_t5 & beam1 & 30 & 0.706 & 0.467 & 8.45 & 0.62 \\

\midrule
\multicolumn{8}{l}{\textbf{Truncation (prefix keep ratio; GPW-SP; $n=40$)}}\\
GPW-SP & truncate\_prefix & 1.00 & 40 & 0.962 & 0.675 & 30.34 & 4.76 \\
GPW-SP & truncate\_prefix & 0.50 & 40 & 0.868 & 0.700 & 13.41 & 2.32 \\
GPW-SP & truncate\_prefix & 0.25 & 40 & 0.675 & 0.325 & 4.72 & 2.20 \\
\midrule
\multicolumn{8}{l}{\textbf{Length scaling (GPW-SP; $n=60$)}}\\
GPW-SP & clean & $L$=40 & 60 & 0.875 & 0.283 & 9.43 & 1.79 \\
GPW-SP & clean & $L$=120 & 60 & 0.960 & 0.733 & 29.29 & 1.63 \\
GPW-SP & clean & $L$=200 & 60 & 0.968 & 0.783 & 47.70 & 2.20 \\
\midrule
\multicolumn{8}{l}{\textbf{Decoding sensitivity (GPW-SP; $n=15$; 120 tokens)}}\\
GPW-SP & clean & $\tau$=0.7 & 15 & 0.929 & 0.467 & 32.97 & -1.67 \\
GPW-SP & clean & $\tau$=0.9 (base) & 15 & 0.991 & 0.933 & 32.64 & 3.72 \\
GPW-SP & clean & $\tau$=1.1 & 15 & 1.000 & 1.000 & 32.31 & 2.06 \\
\midrule
\multicolumn{8}{l}{\textbf{Deletion (GPW-SP; $n=20$)}}\\
GPW-SP & delete & 0.0 & 20 & 0.995 & 0.900 & 33.52 & 8.20 \\
GPW-SP & delete & 0.2 & 20 & 0.970 & 0.600 & 20.95 & 6.38 \\
GPW-SP & delete & 0.4 & 20 & 0.863 & 0.450 & 15.69 & 4.06 \\
\midrule
\multicolumn{8}{l}{\textbf{Mixing / concatenation (GPW-SP; $n=20$; watermarked prefix fraction)}}\\
GPW-SP & mix\_concat & 1.00 & 20 & 0.990 & 0.800 & 32.79 & 8.63 \\
GPW-SP & mix\_concat & 0.75 & 20 & 0.833 & 0.050 & 25.87 & 11.75 \\
GPW-SP & mix\_concat & 0.50 & 20 & 0.210 & 0.000 & 17.65 & 25.04 \\
GPW-SP & mix\_concat & 0.25 & 20 & 0.045 & 0.000 & 10.86 & 29.70 \\
\bottomrule
\end{tabular}
\end{table*}


\end{document}
